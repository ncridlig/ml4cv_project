% chapters/appendixA.tex
\chapter{Technical Details}
\label{app:technical}

\section{Hyperparameter Sweep Configuration}
\label{app:sweep}

% TODO: Include the W&B sweep YAML configuration (1 listing)
% - 13 hyperparameters: lr0, lrf, momentum, weight_decay, warmup_epochs,
%   warmup_momentum, box, cls, dfl, hsv_h, hsv_s, hsv_v, mosaic
% - Bayesian optimization method
% - Include ranges for each parameter

\begin{lstlisting}[style=yaml, caption={W\&B Bayesian sweep configuration.}, label={lst:sweep}]
# TODO: Paste actual sweep YAML from wandb_sweep.yaml
method: bayes
metric:
  name: metrics/mAP50(B)
  goal: maximize
parameters:
  lr0:
    min: 0.001
    max: 0.05
  momentum:
    min: 0.8
    max: 0.98
  # ... (add remaining 11 parameters)
\end{lstlisting}

\section{Two-Stage Training Hyperparameters}
\label{app:two_stage_params}

% TODO: Document exact training parameters for both stages

\begin{table}[ht]
\centering
\caption{Two-stage training configuration.}
\label{tab:two_stage_config}
\begin{tabular}{@{}lll@{}}
\toprule
\textbf{Parameter}  & \textbf{Stage 1 (cone-detector)} & \textbf{Stage 2 (FSOCO-12)} \\ \midrule
Dataset             & cone-detector (22{,}725 imgs)    & FSOCO-12 (5{,}536 imgs)     \\
Epochs              & 338 (converged early)            & 300                         \\
Batch size          & 64                               & 64                          \\
Optimizer           & auto (SGD)                       & AdamW                       \\
Learning rate       & 0.01                             & 0.001                       \\
Image size          & 640                              & 640                         \\
Freeze layers       & None                             & Phase-dependent             \\
Pretrained weights  & COCO (\texttt{yolo26n.pt})       & Stage~1 \texttt{best.pt}    \\ \bottomrule
\end{tabular}
\end{table}

\section{The \texttt{optimizer='auto'} Bug}
\label{app:optimizer_bug}

% TODO: Document the Ultralytics bug encountered during two-stage training (1-2 paragraphs)
% - When optimizer='auto', Ultralytics internally selects SGD or AdamW
%   but IGNORES the user-specified lr0 parameter
% - This caused catastrophic forgetting during Stage 2 fine-tuning:
%   the model's pre-trained features were destroyed in early epochs
% - Diagnosis: loss spiked immediately, mAP50 dropped to near-zero
% - Solution: explicitly set optimizer='AdamW' and lr0=0.001
% - Lesson: always verify optimizer behavior with Ultralytics framework
% - Could include a code snippet showing the fix

\section{fsoco-ubm Dataset Creation Pipeline}
\label{app:ubm_pipeline}

% TODO: Document the complete pipeline for creating the UBM test set
% - Step 1: Convert .mcap ROS bags to AVI videos using ROS2 tools
% - Step 2: Extract frames every 60 frames (2 seconds real-world at 60 FPS)
% - Step 3: Upload to Roboflow workspace (fsae-okyoe/ml4cv_project)
% - Step 4: Annotate with Roboflow Label Assist (5 cone classes)
% - Step 5: Export in YOLO format (critical: use matching format for model version)
% - Important: annotation format must match model (yolo11 vs yolo26 format mismatch bug)

\begin{lstlisting}[style=python, caption={Frame extraction from AVI video.}, label={lst:extraction}]
# TODO: Paste key extract from extract_frames_from_avi.py
import cv2

cap = cv2.VideoCapture("media/lidar1.avi")
frame_count = 0
saved = 0
while cap.isOpened():
    ret, frame = cap.read()
    if not ret:
        break
    if frame_count % 60 == 0:  # every 2 seconds
        cv2.imwrite(f"ubm_test_set/images/lidar1_{saved:04d}.png", frame)
        saved += 1
    frame_count += 1
cap.release()
\end{lstlisting}
